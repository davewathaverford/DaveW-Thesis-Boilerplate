\section{Introduction}

This introduction would normally give an overview of the background,
and explain the purpose of this paper in more detail than the abstract did.
The question being investigated (and its context in the larger field),
and the nature of the answer you found,
should all be clear from this section.
Include as much background as you need here {\em to explain the above},
but if you need a lot more background information to get to your main discussion,
you can also have later section(s) that for detailed background.


Specific results may be described here, and specific papers may be cited.
For example, I'll cite a one-author paper\cite{DuBoulay:Notional86},
a multi-author paper\cite{TFK:Recursion18},
and a paper about CS Education experiments\cite{FTR:Design11}.
Recursion Paper about mental Modeling if recursion \cite{GSG:Recursion03}.

Note how LaTeX pulls things from {\tt CS\_Ed.bib} (due to that name being in my main.tex), 
formats them according to the "alpha" style (also mentioned in main.tex), which gives author names/initials and years,
and then builds the ``biblography'' section and creates the
references here.
